\documentclass[10pt,a4paper,spanish]{report}

\usepackage[light,condensed,math]{iwona}
\usepackage[T1]{fontenc}

\usepackage[spanish]{babel}
\usepackage[utf8]{inputenc}
\usepackage{amsmath, amsthm}
\usepackage{amsfonts, amssymb, latexsym}
\usepackage{enumerate}
\usepackage[official]{eurosym}
\usepackage{graphicx}
\usepackage{graphics}
\usepackage[usenames, dvipsnames]{color}
\usepackage{colortbl}
\usepackage{multicol}
\usepackage{multirow}
\usepackage{fancyhdr}
\usepackage{fancybox}
\usepackage{pseudocode}
\usepackage[all]{xy}
\usepackage{minted}
\usepackage{tikz}
\usepackage{pgfplots}
\usepackage{subfigure}
\usepackage{epigraph}



% \pgfplotsset{compat=1.5}

% a4large.sty -- fill an A4 (210mm x 297mm) page
% Note: 1 inch = 25.4 mm = 72.27 pt
%       1 pt = 3.5 mm (approx)
\topmargin      0 mm    % top margin less 1 inch
\headheight     0 mm    % height of box containing the head
\headsep        10 mm    % space between the head and the body of the page
\textheight     250 mm
\footskip       14 mm    % distance from bottom of body to bottom of foot

% % horizontal page layout -- one inch margin each side
% \oddsidemargin    0   mm    % inner margin less one inch on odd pages
% \evensidemargin   0   mm    % inner margin less one inch on even pages
% \textwidth      159.2 mm    % normal width of text on page


\usepackage[bookmarks=true,
            bookmarksnumbered=false, % true means bookmarks in
                                     % left window are numbered
            bookmarksopen=false,     % true means only level 1
                                     % are displayed.
            colorlinks=true,
            linkcolor=webblue]{hyperref}
\definecolor{webgreen}{rgb}{0, 0.5, 0} % less intense green
\definecolor{webblue}{rgb}{0, 0, 0.5}  % less intense blue
\definecolor{webred}{rgb}{0.5, 0, 0}   % less intense red

\newcommand{\HRule}{\rule{\linewidth}{0.5mm}} % regla horizontal para  el titulo

\pagestyle{fancy}
%con esto nos aseguramos de que las cabeceras de capítulo y de sección vayan en minúsculas

\renewcommand{\sectionmark}[1]{%
      \markright{\thesection\ #1}}
\fancyhf{} %borra cabecera y pie actuales
\fancyhead[LE,RO]{\textcolor[rgb]{0.5,0.8,0.9}{\bfseries\thepage}}
\fancyhead[LO]{\bfseries Braulio Vargas}
\renewcommand{\headrulewidth}{0.5pt}
\renewcommand{\footrulewidth}{0pt}
\addtolength{\headheight}{0.5pt} %espacio para la raya
\fancypagestyle{plain}{%
      \fancyhead{} %elimina cabeceras en páginas "plain"
      \renewcommand{\headrulewidth}{0pt} %así como la raya
}

 \newmintedfile[mycpp]{c++}{
    linenos,
    numbersep=5pt,
    gobble=0,
    frame=lines,
    framesep=2mm,
}

\newmintedfile[mypython]{python}{
    linenos,
    numbersep=5pt,
    gobble=0,
    frame=lines,
    framesep=2mm,
}

%%%%% Para cambiar el tipo de letra en el título de la sección %%%%%%%%%%%
\usepackage{sectsty}
\chapterfont{\fontfamily{pag}\selectfont} %% for chapter if you want
\sectionfont{\fontfamily{pag}\selectfont}
\subsectionfont{\fontfamily{pag}\selectfont}
\subsubsectionfont{\fontfamily{pag}\selectfont}

% Indentacion fuera
\setlength{\parindent}{0pt}
\setlength{\parskip}{1ex plus 0.5ex minus 0.2ex}

\usepackage{titlesec}

\titleformat{\chapter}{\normalfont\huge\center}{--- \thechapter ---}{20pt}{}

\titleformat
{\chapter} % command
[display] % shape
{\Huge\center\bfseries} % format
{--- \thechapter ---} % label
{0.5ex} % sep
{
    \rule{\textwidth}{1pt}
    \vspace{1ex}
    \centering
} % before-code
[
\vspace{-0.5ex}%
\rule{\textwidth}{0.3pt}
] % after-code

\usepackage{cancel}
\usepackage{geometry}
\usepackage{xcolor}
% \usepackage{graphicx}

\definecolor{titlepagecolor}{rgb}{0.0, 0.28, 0.67}
\definecolor{namecolor}{rgb}{0.55, 0.57, 0.67}

%%%%%%%%%%%%%%%%%%%%%%%%%%%%%%%%%%%%%%%%%%%%%%%%%%%%%%%%%%%%%%%%%%%%%%%%%%%%%%%%%%%%%%%%%%%%%%%%%%%%%%%%%%%%%%%%%%%%%%%%%%%%%%%%%%%%%%%%%%%%%%%%%%%%



%%%%%%%%%%%%%%%%%%%%%%%%%%%%%%%%%%%%%%%%%%%%%%%%%%%%%%%%%%%%%%%%%%%%%%%%%%%%%%%%%%%%%%%%%%%%%%%%%%%%%%%%%%%%%%%%%%%%%%%%%%%%%%%%%%%%%%%%%%%%%%%%%%%%

\begin{document}
\begin{titlepage}
\newgeometry{left=7.5cm} %defines the geometry for the titlepage
\pagecolor{titlepagecolor}
\noindent
% \includegraphics[width=2cm]{logo.jpg}\\[-1em]
\color{white}
\makebox[0pt][l]{\rule{1.3\textwidth}{3pt}}
\par
\noindent
\Large{\textbf{\textsf{Guión de Prácticas}} \textcolor{namecolor}{\textsf{Algorítmica}}}\\
\makebox[0pt][l]{\rule{1.3\textwidth}{3pt}}
\vfill
\noindent
{\huge \textsf{Braulio Vargas López}}
\vskip\baselineskip
\noindent
\textsf{\today}
\end{titlepage}
\restoregeometry % restores the geometry
\nopagecolor% Use this to restore the color pages to white
% ----------------------------------------------------------------
\tableofcontents
\newpage

\chapter{\textcolor[rgb]{0.1,0.2,1}Eficiencia \textcolor[rgb]{0.1,0.2,1}Sobre el \textcolor[rgb]{0.1,0.2,1}Código}

\section{\textcolor[rgb]{0.1,0.2,1}Ejercicio 1}

\subsection{\textcolor[rgb]{0.1,0.2,1}Enunciado}

Hallar la eficiencia en el caso promedio del siguiente algoritmo:\\

\begin{minipage}{0.5\textwidth}
\usemintedstyle{rrt}
\begin{minted}
[
linenos,
frame=single,
label={Algoritmo en pseudocódigo},
]
{pascal}
  i:=1
  while i <= n do
    if a[i] >= a[n] then
      a[n]:=a[i]
    end
    i:=i*2
  end
\end{minted}
\end{minipage}
\begin{minipage}{0.5\textwidth}
\begin{minted}
[
frame=single,
label={Algoritmo en C++},
]
{c++}
  int i = 1;
  while (i <= n) {
    if ( a[i] >= a[n] )
      a[n] = a[i];

    i *= 2;
  }
\end{minted}
\end{minipage}

\subsection{\textcolor[rgb]{0.1,0.2,1}Solución}

Para hallar la eficiencia del caso promedio de este algoritmo, haremos uso de las notaciones asintóticas O-grande (en el peor de los casos $\rightarrow O$) y Omega (en el mejor de los casos $\rightarrow \Omega$).

Antes de nada, analizando el algoritmo, podemos ver que hay un bucle $while$, cuya condición de parada es $i \le n$. Lo primero es averiguar cuales son los límites del bucle. Dentro del bucle tenemos una sentencia $if-then$ y una operación $i = i * 2$. Esta operación nos indica que con cada ejecución del bucle, $i$ se incrementa al doble, por lo que el límite del bucle será de orden logarítmico, concretamente $\log_2(n)$, pero como las bases no se tienen en cuenta, se queda como $\log n$.

\begin{displaymath}
  i = 1 \rightarrow 2 \rightarrow 4 \rightarrow \cdots \rightarrow \log_2 n
\end{displaymath}

Ahora que ya sabemos el límite del bucle, vamos a ver cuantas sentencias se repiten en el mejor y peor de los casos:

\begin{description}
  \item [Mejor de los casos]: en el mejor de los casos, $a[i] \ge a[n]$ es cierto en 0 ocasiones, por lo que el número de sentencias que se repiten es:
  \begin{displaymath}
    \underbrace{1}_{i = 1} + \sum_{i = 1}^{\log n} (\underbrace{3}_{if} + \underbrace{2}_{asignaciones} + \underbrace{1}_{while}) = 1 + 6\log n \in \Omega(\log n)
  \end{displaymath}
  \item [Peor de los casos]: en el peor de los casos $a[i] \ge a[n]$ es cierto en todas las ocasiones, por lo que el número de sentencias que se repiten es:
  \begin{displaymath}
    \underbrace{1}_{i = 1} + \sum_{i = 1}^{\log n} (\underbrace{3}_{if} + \underbrace{3}_{then} + \underbrace{2}_{asignaciones} + \underbrace{1}_{while}) = 1 + 9\log n \in O(\log n)
  \end{displaymath}
\end{description}

Una vez hallado los dos casos, a pesar de que coincidan, no podemos asegurar que la eficiencia del caso promedio sea $\Theta(\log n)$. Para hallarlo, tendremos que averiguar e identificar todos los casos posibles. Es decir, desde el mejor de los casos, hasta el peor, pasando por todos los que hay en medio.

Para ello usaremos lo siguiente:

\begin{displaymath}
  t(n) = \sum_{j=0}^{\log n} \text{Prob}\left[\underbrace{a[i]\ge a[n]}_{\text{cierto en }j\text{ ocasiones}}\right] \cdot t\left[\underbrace{a[i]\ge a[n]}_{\text{no cierto en }j\text{ ocasiones}}\right]
\end{displaymath}

Como no podemos calcular la probabilidad de todos los casos, se usará el enfoque de máxima verosimilitud, en el que la probabilidad de ser cierto o no será igual para todos los casos, es decir, todos los casos son equiprobables, simplificando el modelo y facilitando el ajuste.
\begin{center}
\begin{displaymath}
  \sum_{j=0}^{\log n} \frac{1}{\log(n) + 1} \cdot \left [ \underbrace{1}_{i=1} + \underbrace{\sum_{k = 1}^{j}(1 + 3 + 3 + 2)}_{\text{cierto en }j\text{ ocasiones}} + \underbrace{\sum_{k = j + 1}^{\log n}(1 + 3 + 2)}_{\text{no cierto en }j\text{ ocasiones}} \right]
\end{displaymath}

A continuación, vamos a sumar los términos y resolver las sumatorias:
\begin{displaymath}
  \sum_{j=0}^{\log n} \frac{1}{\log(n) + 1}\cdot\left[ 1 + 9j + 6(\log n - j) \right] \quad = \quad \sum_{j=0}^{\log n} \frac{1}{\log(n) + 1}\cdot\left[ 6\log n + 3j + 1 \right]
\end{displaymath}

Como $\frac{1}{\log n + 1}$ no depende de ningún término, lo sacamos fuera de la sumatoria y dividimos la sumatoria por términos:
\begin{displaymath}
  \frac{1}{\log n + 1} \sum_{j = 0}^{\log n} 6\log n + 3j + 1 \quad = \quad \frac{1}{\log n + 1} \left(\sum_{j = 0}^{\log n} 6\log n + \sum_{j = 0}^{\log n} 3j + \sum_{j = 0}^{\log n} 1 \right)
\end{displaymath}

Ahora resolvemos las sumatorias y combinamos los resultados:
\begin{displaymath}
  \frac{1}{\log n + 1} \left(6\log n \cdot \log n + \frac{3(\log n + 1)\log n}{2} + (\log n + 1) \right)
\end{displaymath}
\begin{displaymath}
  \frac{6(\log n)^2}{\log n + 1} + \frac{3(\log n + 1)\log n}{2(\log n + 1)} + \frac{\log n + 1}{\log n + 1}
\end{displaymath}

El siguiente paso es reunir todo en la misma fracción y quitar logaritmos:
\begin{displaymath}
  \frac{12(\log n)^2 + 3(\log n)^2 + 3\log n + 2\log n + 2}{2(\log n + 1)} \quad = \quad \frac{15(\log n)^2 + 5\log n + 2}{2(\log n + 1)} \quad = \quad 
  \frac{\frac{15(\log n)^2}{\log n} + \frac{5\log n}{\log n} + \frac{2}{\log n}}{\frac{2(\log n + 1)}{\log n}}
\end{displaymath}
\begin{displaymath}
  \frac{13\log n + 3 + \frac{2}{\log n}}{2 + \frac{2}{\log n}}
\end{displaymath}

Para hallar la eficiencia en el caso promedio, calculamos el límite cuando n tiende a $\infty$:
\begin{displaymath}
  \lim_{n\rightarrow \infty} \frac{15\log n + 5 + \frac{2}{\log n}}{2 + \frac{2}{\log n}} \quad \Rightarrow \quad \lim_{n\rightarrow \infty} \frac{15\log n + 5 + \cancel{\frac{2}{\log n}}}{2 + \cancel{\frac{2}{\log n}}} \quad 
\end{displaymath}
\begin{displaymath}
 \lim_{n\rightarrow \infty} \frac{15 \log n + 5}{2} \quad \Rightarrow \quad \lim_{n\rightarrow \infty} \log n \in \Theta(\log n)
\end{displaymath}
\end{center}
Como resultado, obtenemos que la eficiencia en el caso promedio es $\mathbf{\Theta(\log n)}$.

\newpage

\chapter{\textcolor[rgb]{0.1,0.2,1}Eficiencia \textcolor[rgb]{0.1,0.2,1}Teórica}

\subsection{\textcolor[rgb]{0.1,0.2,1}Enunciado}

Obtener el orden de eficiencia de la siguiente ecuación recursiva:

\begin{equation*}
t(n) = 
\begin{cases}
1 & \textit{si } n = 1, \\
3t(\frac{n}{2}) + n & \textit{si } n > 1
\end{cases}
\end{equation*}

\subsection{\textcolor[rgb]{0.1,0.2,1}Solución}

\begin{center}
  Para empezar, vamos a quedarnos con la fórmula general para resolver la ecuación:

  \begin{displaymath}
    t(n) = 3t\left(\frac{n}{2}\right) + n
  \end{displaymath}

  Como vemos, la ecuación general es una ecuación lineal no homogénea y lo que haremos a continuación será obtener las raíces de la parte homogénea y la parte no homogénea:

  \begin{displaymath}
    t(n) - 3t\left(\frac{n}{2}\right) = n
  \end{displaymath}
  Para obtener las raíces de la ecuación, haremos el siguente cambio de variable para hacer los cálculos más sencillos:
  \begin{displaymath}
    n = 2^k \qquad \Longrightarrow \quad k = \log_2 n
  \end{displaymath}
  Ahora la ecuación queda así:
  \begin{displaymath}
    t(2^k) - 3t(2^{k-1}) = 2^k
  \end{displaymath}
  Ahora vamos a obtener las raíces de la parte homogénea, olvidándonos de la parte no homogénea de la ecuación ($2^k$):
  \begin{displaymath}
    t(2^k) -3t(2^{k-1}) = 0 \longrightarrow t_k - 3\cdot t_{k-1} = 0 \longrightarrow x - 3 = 0
  \end{displaymath}

  Las ráices de la parte homogénea son $(x-3)$. Ahora, vamos a calcular las raíces de la parte no homogénea, usando el método de la ecuación característica: $a_0x^n + a_1x^{n-1}+\ldots+a_nx^0 = b^n\cdot p(n)$.

  \begin{displaymath}
      b^n\cdot p(n) \Rightarrow 2^k(1)
  \end{displaymath}

  Como en este caso, el polinomio es de grado 0 y $b^n = 2^k$, las raíces de la parte homogénea son $(x-2)$. En conjunto, la ecuación tiene las raíces $(x-3)(x-2) = 0$.A continuación, se deshace el cambio de variable sustituyendo $k$ por $\log_2 n$ quedando lo siguiente:
  \begin{displaymath}
    t(n) = c_1\cdot 3^k + c_22^k
  \end{displaymath}
  \begin{displaymath}
    t(n) = c_1\cdot 3^{\log_2n} + c_22^{\log_2n} \longrightarrow c_1\left(3^{\left(\frac{\log_3n}{\log_32}\right)}\right) + c_2n
  \end{displaymath}
  \begin{displaymath}
    t(n) = c_1\left(3^{\log_3n}\right)^{\left(\frac{1}{\log_32}\right)} + c_2n
  \end{displaymath}
  \begin{displaymath}
    t(n) = c_1\left(n\right)^{\left(\frac{1}{\log_32}\right)} + c_2n
  \end{displaymath}
\end{center}

Una vez resuelta y reducida la ecuación, tenemos que hayar el orden de eficiencia de la ecuación. Siguiendo un primer impulso se puede decir que el orden de eficiencia es $O\left(n\right)^{\left(\frac{1}{\log_32}\right)}$ ya que es el término que crece antes hacia el infinito. Pero hay que tener el cuenta el valor de las constantes que acompañan a cada miembro. Esto es así, porque en caso de que el valor de $c_1$ sea menor o igual que cero, el tiempo que se obtiene es negativo cosa que en la práctica es imposible. Vamos a comprobar el valor de $c_1$ y $c_2$, pero con la peculiaridad de que no podemos usar el caso base de la ecuación ya que no representa el tiempo de ejecución real para nuestro caso base, ya que este tiempo depende de factores externos al algoritmo, como pueden ser el sistema operativo que ejecute el algoritmo, las características de la máquina, de los datos$\ldots$, Por lo que para obtener si las constastes son positivas o negativas, lo haremos sustituyendo las soluciones en la ecuación general:
\begin{center}
\begin{displaymath}
  n = t(n) - 3t\left(\frac{n}{2}\right)
\end{displaymath}
Teniendo en cuenta que $\left(\frac{1}{2}\right) = \frac{1}{3}$:
\begin{displaymath}
  n = t(n) - 3t\left(\frac{n}{2}\right)
\end{displaymath}
\begin{displaymath}
  n = \left(\cancel{c_1n^{\left(\frac{1}{\log_32}\right)}} + c_2n\right) - 3t\left(\cancel{c_1n^{\left(\frac{1}{\log_32}\right)}} + c_2n\right)
\end{displaymath}
\begin{displaymath}
  n = c_2n -3c_2\frac{n}{2} \quad = \quad c_2n\left(1 - \frac{3}{2}\right) \quad = \quad -c\frac{n}{2}
\end{displaymath}
\begin{displaymath}
  n = -c\frac{n}{2} \longrightarrow c_2 = -2
\end{displaymath}
\end{center}

Como hemos podido comprobar, $c_2$ es negativo, por lo que el valor de $c_1$ tiene que ser positivo, ya que de no ser así, el tiempo de ejecución sería cada vez menor hasta que pasara a ser negativo, cosa que en la práctica es imposible. Por lo que el orden de eficiencia de esta ecuación es $O(n^{\log_23})$

\end{document}